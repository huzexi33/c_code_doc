% arara: xelatex
\documentclass[a4paper,UTF8]{ctexart}

\usepackage{geometry}
\geometry{
  % showframe,
  headheight  = 0.6cm,
  headsep     = 0.6cm,
  footskip    = 1.2cm,
  inner       = 2.0cm,
  outer       = 2.0cm,
  top         = 2.8cm,
  bottom      = 2.8cm,
  nomarginpar,
}

\usepackage{calc}
\usepackage{multicol,booktabs}
\usepackage{pstricks,listings}
\usepackage{enumitem}
\usepackage{xcolor}
\usepackage{array}
\usepackage{longtable}
\usepackage{tabularx}
\usepackage{makecell}
\definecolor{OceanBlue}{RGB}{0,77,128}
\usepackage{amsmath}
\usepackage{physics}
\usepackage{siunitx}
\sisetup{
  number-math-rm = \mathnormal,
  inter-unit-product=\,,
  per-mode = reciprocal,
  separate-uncertainty = false,
  range-units = single
}
\usepackage{fontspec}
\usepackage{unicode-math}
\unimathsetup{
  math-style = ISO,
  bold-style = ISO,
  nabla      = upright,
  partial    = italic,
}
\setmainfont{lmroman10}[
  Extension      = .otf,
  UprightFont    = *-regular,
  BoldFont       = *-bold,
  ItalicFont     = *-italic,
  BoldItalicFont = *-bolditalic,
]%
\setsansfont{lmsans10}[
  Extension      = .otf,
  UprightFont    = *-regular,
  BoldFont       = *-bold,
  ItalicFont     = *-oblique,
  BoldItalicFont = *-boldoblique,
]%
\setmonofont{lmmonolt10}[
  Extension      = .otf,
  UprightFont    = *-regular,
  BoldFont       = *-bold,
  ItalicFont     = *-oblique,
  BoldItalicFont = *-boldoblique,
  ]%
\setmathfont{latinmodern-math.otf}%

\usepackage{newunicodechar}
\newunicodechar{ff}{ff}
\newunicodechar{fi}{fi}
\newunicodechar{fl}{fl}
\newunicodechar{ffi}{ffi}
\newunicodechar{ffl}{ffl}

\setCJKmainfont{Songti SC}[
  UprightFont    = * Regular,
  % UprightFont    = * Light,
  BoldFont       = * Bold,
  % BoldFont       = * Black,
  ItalicFont     = Kaiti SC Regular,
  BoldItalicFont = Kaiti SC Bold,
]%
\setCJKsansfont{Heiti SC}[BoldFont=* Medium]%
\setCJKmonofont{STFangsong}
\setCJKfamilyfont{zhsong}{Songti SC}[
  UprightFont = * Regular,
  BoldFont  = * Bold,
]%
\setCJKfamilyfont{zhhei}{Heiti SC}[
  UprightFont = * Light,
  BoldFont    = * Medium,
]%
\setCJKfamilyfont{zhfs}{STFangsong}%
\setCJKfamilyfont{zhkai}{Kaiti SC}[BoldFont = * Bold]%
\setCJKfamilyfont{zhli}{Baoli SC}%
\setCJKfamilyfont{zhyuan}{Yuanyi SC}[
  UprightFont = * Light,
  BoldFont    = * Bold,
]%
\everymath{\displaystyle}

\setmathfont{STIX2Math}[
    Extension    = .otf,
    StylisticSet = 8,
    range="25B1,
]%

\usepackage{xcolor}
\definecolor{codegreen}{rgb}{0,0.6,0}
\definecolor{codegray}{rgb}{0.5,0.5,0.5}
\definecolor{codepurple}{rgb}{0.58,0,0.82}
\definecolor{backcolour}{rgb}{0.95,0.95,0.92}
\usepackage{listings}
\lstdefinestyle{mystyle}{
    backgroundcolor=\color{backcolour},   
    commentstyle=\color{codegreen},
    keywordstyle=\color{magenta},
    numberstyle=\color{codegray},
    stringstyle=\color{codepurple},
    basicstyle=\ttfamily,
    breakatwhitespace=false,         
    breaklines=true,                 
    captionpos=b,                    
    keepspaces=true,                 
    numbers=left,                    
    numbersep=5pt,                  
    showspaces=false,                
    showstringspaces=false,
    showtabs=false,                  
    tabsize=2
}
\lstset { %
  language=C++,
  style=mystyle,
  numbers=none,
    % backgroundcolor=\color{black!5}, % set backgroundcolor
    % keywordstyle={\ttfamily\color{blue}},
    % basicstyle={\ttfamily\color},% basic font setting
}

\begin{document}

\pagestyle{plain}
\setlength{\parindent}{0pt}

% \title{课件习题汇总}
% \date{}
% \maketitle
% \vspace{-2cm}

\setcounter{section}{2}
\section{Lecture 3}

\subsection{基本输入输出操作}
\begin{itemize}[leftmargin=0pt, rightmargin=0cm, labelwidth=0.8cm, labelsep=0.2cm]
\item Insertion operator: \lstinline{std::cout} 与 extraction operator
  \lstinline{std::cin}.
\item 格式化输出需要 \lstinline{#include <iomanip>}. \lstinline{std::setw()} 用
  于设置输出宽度. \lstinline{std::setprecision()} 设置保留的有效位数, 与
  \lstinline{std::fixed} 组合设置保留小数点后的位数. \lstinline{std::setiosflags(std::ios::left)} 使输出左对齐.
\item 转义字符、脱字符 \lstinline{\n \' \\ \b \t \" \?} 等.
\end{itemize}

\subsection{C++ 的简单计算}
\begin{itemize}[leftmargin=0pt, rightmargin=0cm, labelwidth=0.8cm, labelsep=0.2cm]
\item 算数算符 \lstinline{+ - * /} 等.
\item 常量修饰符, 也是 C++ 的一个关键词 \lstinline{const}.
\item 求余算符 \lstinline{%}, 表达式必须为整数类型.
\item 增减算符 \lstinline{++}, \lstinline{--}, 以及 \lstinline{+= -= *= /=} 等.
\item 算符的优先级(precedence 小的优先, 对于相同的 precedence, associativity 表
  示运算优先的一侧):
  \vspace{-5pt}
  \begin{center}
  \begin{tabularx}{\textwidth}{| X | l | X |}
  \hline
   precedence & operators & associativity \\ \hline
   1 & \lstinline$::$ & None \\ \hline
   2 & \lstinline$()$ \quad \lstinline$[]$ \quad \lstinline$->$ \quad
   \lstinline$.$ \quad \lstinline$typeid$ \quad 后缀 \lstinline$++$ \quad
   \lstinline$--$ & Left \\
   & \lstinline$const_cast$ \quad \lstinline$dynamic_cast$ \quad
   \lstinline$static_cast$ \quad \lstinline$reinterpret_cast$ & \\ \hline
   3 & 逻辑否 \lstinline$!$ \quad 补 \lstinline$\~$ \quad 一元 \lstinline$+$
   \quad \lstinline$-$ \quad 前缀 \lstinline$++$ \quad \lstinline$--$ & Right \\
   & 取地址 \lstinline$\&$ \quad 取值 \lstinline$*$ \quad 旧式类型转换
   \lstinline$(type)$ & \\
   & \lstinline$sizeof$ \quad \lstinline$decltype$ & \\
   & \lstinline$new$ \quad \lstinline$new[]$ \quad \lstinline$delete$ \quad
   \lstinline$delete[]$ & \\ \hline
   4 & \lstinline$.*$ \quad \lstinline$->*$ & Left \\ \hline
   5 & \lstinline$*$ \quad \lstinline$/$ \quad \lstinline$\%$ & Left \\ \hline
   6 & \lstinline$+$ \quad \lstinline$-$ & Left \\ \hline
   7 & \lstinline$\<<$ \quad \lstinline$\>>$ & Left \\ \hline
   8 & \lstinline$==$ \quad \lstinline$!=$ & Left \\ \hline
   9 & \lstinline$\&$ & Left \\ \hline
   10 & \lstinline$\^$ & Left \\ \hline
   11 & \lstinline$|$ & Left \\ \hline
   12 & \lstinline$\&\&$ & Left \\ \hline
   13 & \lstinline$||$ & Left \\ \hline
   14 & \lstinline$?:$ & Right \\ \hline
   15 & \lstinline$=$ \quad \lstinline$*=$ \quad \lstinline$/=$ \quad
   \lstinline$\%=$ \quad \lstinline$+=$ \quad \lstinline$-=$ \quad
   \lstinline$\&=$ \quad \lstinline$\^=$ \quad \lstinline$|=$ \quad
   \lstinline$\<<=$ \quad \lstinline$\>>=$ & Right \\ \hline
   16 & \lstinline$throw$ & Right \\ \hline
   17 & \lstinline$,$ & Left \\ \hline
  \end{tabularx}
  \end{center}
  \vspace{-5pt}
\item 类型转换 (converting \& casting). \\
  隐式类型转换: 不同类型变量之间的运算, 将高阶的降为低阶来运算(若负的有符号变量降阶为无符号变量会出错):
  \vspace{-5pt}
  \begin{center}
  \begin{tabularx}{\textwidth}{| X | X | X |}
  \hline
   1. \lstinline$long double$ & 2. \lstinline$double$ & 3. \lstinline$float$
   \\ \hline
   4. \lstinline$unsigned long long$ & 5. \lstinline$long long$ & \\ \hline
   6. \lstinline$unsigned long$ & 7. \lstinline$long$ & \\ \hline
   8. \lstinline$unsigned int$ & 9. \lstinline$int$ & 10. \lstinline$short int$ \\ \hline
   11. \lstinline$char$ & 12. \lstinline$bool$ & \\ \hline
  \end{tabularx}
  \end{center}
  \vspace{-5pt}
  显式类型转换: \lstinline{static_cast<type>(expression)}.
\item \lstinline{auto} 关键字.
\item 使用 \lstinline{typeid(expression).name()} 获取表达式类型.
\end{itemize}

\subsection{位运算}

\begin{itemize}[leftmargin=0pt, rightmargin=0cm, labelwidth=0.8cm, labelsep=0.2cm]
\item 6 个位算符的运算规则(这里的几元运算指有几个二进制数参与运算):
\vspace{-5pt}
\begin{center}
\begin{tabularx}{\textwidth}{| l | X | X | X |}
\hline
二元 & \lstinline$\&$ 位与: 有 0 为 0, 否则为 1 & \lstinline$|$ 位或: 有 1 为 1, 否则为
0 & \lstinline$\^$ 位异或: 相同为 0, 不同为 1 \\ \hline
一元 & \lstinline$\~$ 取反  & \lstinline$\<<$ 左移: 左移 $n$ 位,
末端补 0 & \lstinline$\>>$ 右移: 右移 $n$ 位, 末端补 0 \\ \hline
\end{tabularx}
\end{center}
\vspace{-5pt}
\end{itemize}

\section{Lecture 4}

\subsection{左值与右值 (lvalue \& rvalue)}

\begin{itemize}[leftmargin=0pt, rightmargin=0cm, labelwidth=0.8cm, labelsep=0.2cm]
\item 左值: 持续性存储的数据.
\item 右值: 暂时性存储的表达式结果.
\end{itemize}

\subsection{存储时间与作用域}

\begin{itemize}[leftmargin=0pt, rightmargin=0cm, labelwidth=0.8cm, labelsep=0.2cm]
\item 存储时间:\\
  (1) 自动存储时间 (automatic): 在代码块作用域(一堆大括号内)内声明的局部变量, 会
  自动分配内存空间, 离开作用域后会自动归还内存空间. \\
  (2) 静态存储时间 (static): 在文件或命名空间作用域声明的变量, 以及在代码块作用
  域内用 \lstinline{static} 标识的变量, 它的内存空间会在整个程序运行的时间被它占
  用. \\
  (3) 动态存储时间 (dynamic): 用 \lstinline{new} 和 \lstinline{delete} 申请和归
  还内存空间的变量, 在 \lstinline{new} 与 \lstinline{delete} 之间持续存在.
\end{itemize}

\subsection{枚举}

通常 \lstinline{enumerator} 为 \lstinline{int} 类型, 如果不初始化, 第一个的值为 0, 之后的值比前面多 1.
\begin{itemize}[leftmargin=0pt, rightmargin=0cm, labelwidth=0.8cm, labelsep=0.2cm]
\item 旧式(大括号不表示作用域, 相当于 \lstinline{#define enumerator (constexpr)}):\\
  \lstinline$enum name(optional) { enumerator = constexpr , enumerator = constexpr , ... };$ \\
  \lstinline$enum name(optional) : type { enumerator = constexpr , enumerator = constexpr , ... };$
\begin{lstlisting}
enum color { red, yellow, green = 20, blue };
color col = red;
int n = blue; // n == 21
\end{lstlisting}
\item 新式(大括号表示作用域, 调用时使用 ):\\
  \lstinline$enum class name(optional) { enumerator = constexpr , enumerator = constexpr , ... };$ \\
  \lstinline$enum class name(optional) : type { enumerator = constexpr , enumerator = constexpr , ... };$
\begin{lstlisting}
enum class Color { red, green = 20, blue };
Color r = Color::blue;
switch(r)
{
    case Color::red  : std::cout << "red\n";   break;
    case Color::green: std::cout << "green\n"; break;
    case Color::blue : std::cout << "blue\n";  break;
}
// int n = r; // error: no implicit conversion from scoped enum to int
int n = static_cast<int>(r); // OK, n = 21
\end{lstlisting}
\item 命名空间:\\
(1) \lstinline{using namespace std;} \quad 可以使用所有头文件中包含的
\lstinline{std} 中的内容, 使用时可省略 \lstinline{std::}. \\
(2) \lstinline{using std::cout;} \quad 只使用 \lstinline{std::cout}, 使用时只有
它省略 \lstinline{std::}.
\end{itemize}

\subsection{命名空间}

\begin{itemize}[leftmargin=0pt, rightmargin=0cm, labelwidth=0.8cm, labelsep=0.2cm]
\item 命名空间的用法:\\
(1) \lstinline{using namespace std;} \quad 可以使用所有头文件中包含的
\lstinline{std} 中的内容, 使用时可省略 \lstinline{std::}. \\
(2) \lstinline{using std::cout;} \quad 只使用 \lstinline{std::cout}, 使用时只有
它省略 \lstinline{std::}.
\item 声明命名空间: \lstinline$namespace ns_name { declarations };$
\begin{lstlisting}
#include <vector>
 
namespace vec {
    template< typename T >
    class vector {
        // ...
    };
} // of vec
 
int main()
{
    std::vector<int> v1; // Standard vector.
    vec::vector<int> v2; // User defined vector.
    v1 = v2;             // Error: v1 and v2 are different object's type.
    {
        using namespace std;
        vector<int> v3;  // Same as std::vector
        v1 = v3; // OK
    }
    {
        using vec::vector;
        vector<int> v4;  // Same as vec::vector
        v2 = v4; // OK
    }
    return 0;
}
\end{lstlisting}
\item 多次声明一个命名空间(名字相同): 所有的东西都在是同一个命名空间. 例如
  \lstinline{<iostream>} 中的 \lstinline{std} 和 \lstinline{<string>} 中的 \lstinline{std}.
\end{itemize}

\section{Lecture 5: 判断和循环}

\subsection*{比较变量}

\begin{itemize}[leftmargin=0pt, rightmargin=0cm, labelwidth=0.8cm, labelsep=0.2cm]
\item 6 个比较运算符: \lstinline{<} \quad \lstinline{>} \quad \lstinline{==}
  \quad \lstinline{<=} \quad \lstinline{>=} \quad \lstinline{!=}
\item 逻辑运算符: \lstinline{\&\&} \quad \lstinline{||} \quad \lstinline{!}
\item \lstinline{if} 语句:
\begin{lstlisting}
if(boolean_expression 1) {
   // Executes when the boolean expression 1 is true
} else if( boolean_expression 2) {
   // Executes when the boolean expression 2 is true
} else if( boolean_expression 3) {
   // Executes when the boolean expression 3 is true
} else {
   // executes when the none of the above condition is true.
}
\end{lstlisting}
\item 条件运算符(三元运算符): \lstinline{condition ? expression 1 : expression 2;}
\item \lstinline{switch} 语句:\\
(1) 如果没有 \lstinline{break}, 程序在执行相应的 \lstinline{case} 后的语句后, 不
会跳出大括号, 而是继续往下执行, 直到遇到 \lstinline{break} 或末尾的大括号.
(2) 所有 \lstinline{constant-expression} 必须不同, 且必须是编译时可确定的整数值.
\begin{lstlisting}
switch(expression) {
   // all the constant-expression must have distinct values.
   case constant-expression 1 :
      statement(s);
      break; //optional
   case constant-expression 2 :
      statement(s);
      break; //optional
   // you can have any number of case statements.
   default : //Optional
      statement(s);
}
\end{lstlisting}
\item \lstinline{goto} 语句(不建议使用):
\begin{lstlisting}
Syntax1      |   Syntax2
----------------------------
goto label;  |    label:  
.            |    .
.            |    .
.            |    .
label:       |    goto label;
\end{lstlisting}
\end{itemize}

\section{Lecture 6: 循环语句}

\begin{itemize}[leftmargin=0pt, rightmargin=0cm, labelwidth=0.8cm, labelsep=0.2cm]
\item \lstinline{for} 语句:\\
  (1) 在 \lstinline{statement} 内的 \lstinline{break} 语句会使程序直接跳出
  \lstinline{for} 循环体.\\
  (2) 在 \lstinline{statement} 内的 \lstinline{continue} 语句会使程序略过
  \lstinline{statement} 内其他语句执行 \lstinline{iteration_expression}, 再判断
  \lstinline{condition}, \dots \\
  (3) \lstinline{break} 和 \lstinline{continue} 语句都只跳出它所在最里层的循环.\\
  (4) \lstinline{condition} 为空, 等价于 \lstinline{condition} 为 \lstinline{true}.
\begin{lstlisting}
for ( init-statement ; condition ; iteration_expression ) { statement; }
\end{lstlisting}
\item \lstinline{while} 语句:\\
如果 \lstinline{condition} 是形如 \lstinline{type var = x} 的声明语句, 则这个变
量 \lstinline{var} 在每次循环时都会销毁和重建.
\begin{lstlisting}
while ( condition ) { statement; }
\end{lstlisting}
\item \lstinline{do-while} 语句(注意不要忘记最后的分号):
\begin{lstlisting}
do { statement; } while ( condition ) ;
\end{lstlisting}
\item 基于范围的 \lstinline{for} 循环:
\begin{lstlisting}
for ( range_declaration : range_expression ) loop_statement;
\end{lstlisting}
\begin{lstlisting}
#include <iostream>
#include <string>
int main() 
{
    // the initializer may be a braced-init-list
    for (int n : {0, 1, 2, 3, 4, 5})
        std::cout << n << ' ';
    std::cout << '\n';
    // Iterating over array
    int a[] = {0, 1, 2, 3, 4, 5};     
    for (int n : a)
        std::cout << n << ' ';
    std::cout << '\n';
    // Printing string characters
    std::string str = "Geeks";
    for (char c : str) 
        std::cout << c << ' ';
    std::cout << '\n';
}
\end{lstlisting}
\end{itemize}

\section{Lecture 7, 8, 9, 10, 11: 数组、字符串和指针}

\subsection{数组}

\begin{itemize}[leftmargin=0pt, rightmargin=0cm, labelwidth=0.8cm, labelsep=0.2cm]
\item 数组的声明: \lstinline{type arrayName [ arraySize ];} 
\begin{lstlisting}
double array[10];
// length of initializer list must not exceed array length.
double array[5] = {1000.0, 2.0, 3.4, 17.0, 50.0};
double array[] = {1000.0, 2.0, 3.4, 17.0, 50.0};
double array[5] = {}; // set all elements to 0.
double array[5] = {0.5}; // set the first element to 0.5 and the rest to 0.
\end{lstlisting}
\item 数组的结构:
  \vspace{-5pt}
  \begin{center}
  \begin{tabularx}{\textwidth}{| l | X | X | X | X | X |}
  \hline
   index & 0 & 1 & 2 & 3 & 4 \\ \hline
   position & \makecell[l]{\lstinline$array + 0$ \\ \lstinline$\&array[0]$} &
   \makecell[l]{\lstinline$array + 1$ \\ \lstinline$\&array[1]$} &
   \makecell[l]{\lstinline$array + 2$ \\ \lstinline$\&array[2]$} &
   \makecell[l]{\lstinline$array + 3$ \\ \lstinline$\&array[3]$} &
   \makecell[l]{\lstinline$array + 4$ \\ \lstinline$\&array[4]$} \\ \hline
   % value & \lstinline$*(array + 0)$ \qquad \lstinline$array[0]$ & \lstinline$*(array
   % + 1)$ \qquad \lstinline$array[1]$ & \lstinline$*(array + 2)$ \qquad
   % \lstinline$array[2]$ & \lstinline$*(array + 3)$ \qquad \lstinline$array[3]$ &
   % \lstinline$*(array + 4)$ \qquad \lstinline$array[4]$ \\ \hline
  \end{tabularx}
  \end{center}
  \vspace{-5pt}
\item 二维数组的声明:
\begin{lstlisting}
int array[3][5];
int array[3][5] = {};
int a[3][5] = {  
   {0, 1, 2, 3, 4} ,      /*  initializers for row indexed by 0 */
   {5, 6, 7, 8, 9} ,      /*  initializers for row indexed by 1 */
   {10, 11, 12, 13, 14}   /*  initializers for row indexed by 2 */
};
// equivalent to
int a[3][5] = {0,1,2,3,4,5,6,7,8,9,10,11,12,13,14};
int a[3][5] = {  
   {0, 1, 2} ,            /*  rest unspecified elements are 0 */
   {5, 6, 7, 8, 9} ,
   {10, 11, 12, 13}       /*  rest unspecified elements are 0 */
};
\end{lstlisting}
\item 二维数组的结构(地址):
  \vspace{-5pt}
  \begin{center}
  \begin{tabularx}{\textwidth}{| l | X | X | X | X | X |}
  \hline
   index & 0 & 1 & 2 & 3 & 4 \\ \hline
   0 & \makecell[l]{\lstinline$array+0$ \\ \lstinline$*(array+0)+0$ \\ \lstinline$array[0]$} &
   \makecell[l]{\lstinline$*(array+0)+1$ \\ \lstinline$array[0]+1$} &
   \makecell[l]{\lstinline$*(array+0)+2$ \\ \lstinline$array[0]+2$} &
   \makecell[l]{\lstinline$*(array+0)+3$ \\ \lstinline$array[0]+3$} &
   \makecell[l]{\lstinline$*(array+0)+4$ \\ \lstinline$array[0]+4$} \\ \hline
   1 & \makecell[l]{\lstinline$array+1$ \\ \lstinline$*(array+1)+0$ \\ \lstinline$array[1]$} &
   \makecell[l]{\lstinline$*(array+1)+1$ \\ \lstinline$array[1]+1$} &
   \makecell[l]{\lstinline$*(array+1)+2$ \\ \lstinline$array[1]+2$} &
   \makecell[l]{\lstinline$*(array+1)+3$ \\ \lstinline$array[1]+3$} &
   \makecell[l]{\lstinline$*(array+1)+4$ \\ \lstinline$array[1]+4$} \\ \hline
   2 & \makecell[l]{\lstinline$array+2$ \\ \lstinline$*(array+2)+0$ \\ \lstinline$array[2]$} &
   \makecell[l]{\lstinline$*(array+2)+1$ \\ \lstinline$array[2]+1$} &
   \makecell[l]{\lstinline$*(array+2)+2$ \\ \lstinline$array[2]+2$} &
   \makecell[l]{\lstinline$*(array+2)+3$ \\ \lstinline$array[2]+3$} &
   \makecell[l]{\lstinline$*(array+2)+4$ \\ \lstinline$array[2]+4$} \\ \hline
  \end{tabularx}
  \end{center}
  \vspace{-5pt}
\end{itemize}

\subsection{C 式字符串}

\begin{itemize}[leftmargin=0pt, rightmargin=0cm, labelwidth=0.8cm, labelsep=0.2cm]
\item \lstinline{char} 类型数组可用于存储 C 式字符串, 字符序列后必须有一个空字符
  \lstinline{'\0'}.
\item 声明(必须留有一位保存 \lstinline{'\0'}):
\begin{lstlisting}
char city[11] = "Guang Dong";
char city[] = "Guang Dong";
\end{lstlisting}
\item 字符串输入:
\begin{lstlisting}
const int MAX = 80;
char name[MAX];
cin.getline(name, MAX, '\n');
\end{lstlisting}
\item 二维字符数组存储多个 C 式字符串:
\begin{lstlisting}
char stars[6][80] = { "Robert Redford",
                      "Hopalong Cassidy",
                      "Lassie",
                      "Slim Pickens",
                      "Boris Karloff",
                      "Oliver Hardy"
                    };
\end{lstlisting}
\end{itemize}

\subsection{间接数据访问(指针)}

\begin{itemize}[leftmargin=0pt, rightmargin=0cm, labelwidth=0.8cm, labelsep=0.2cm]
\item 指针是存储给定类型变量地址的变量.
\item 声明:
\begin{lstlisting}
long* pnumber;
long* pnumber, number{99};
long* pnumber = &number;
\end{lstlisting}
\item 引用运算符(取地址运算符) \lstinline{\&} \quad 解引用运算符(间接寻址运算符)
  \lstinline{*}.
\item 区别 \lstinline{*} 的三种用途:
\begin{lstlisting}
a * b;  // multiply operator
int *p; // pointer definition symbol
a = *p; // dereference operator
\end{lstlisting}
\item 指向 \lstinline{char} 的指针(双引号内的字符串成为字符字面量, 前面的指针最
  好加上修饰符 \lstinline{const}):
\begin{lstlisting}
const char *proverb {"A miss is as good as a mile."};
\end{lstlisting}
\item 指针与 \lstinline{const} (判断标准为 \lstinline{const} 与 \lstinline{*} 的
  次序):\\
(1) \lstinline{const} 位于 \lstinline{*} 之前: \lstinline{const int * p = &a;}
与 \lstinline{int const * p = &a;} 等价, 都表示 \lstinline{p} 指向变量的值不能通过
\lstinline{p} 改变, 但 \lstinline{p} 可以指向别处.\\
(2) \lstinline{const} 位于 \lstinline{*} 之后: \lstinline{int * const p = &a;},
表示 \lstinline{p} 不能指向别处, 但可以通过 \lstinline{p} 改变指向变量的值.
(3) \lstinline{*} 左右都有 \lstinline{const}: 既不能改变 \lstinline{p} 的指向,
也不能通过 \lstinline{p} 改变所指变量的值.
注意: 变量的值是可以通过其他方式改变的, 这里只是讲通过指针能否改变.
\end{itemize}

\subsection{动态内存分配}

\begin{itemize}[leftmargin=0pt, rightmargin=0cm, labelwidth=0.8cm, labelsep=0.2cm]
\item 动态存储期的变量使用的内存空间称为堆 (heap) 或自由存储区 (free store).
\item \lstinline{new} 与 \lstinline{delete} 算符(多维数组的
  \lstinline{delete} 后仍只有一个 \lstinline{[]}):
\begin{lstlisting}
pointer = new type;
pointer = new type [number_of_elements];
delete pointer;
delete[] pointer;
\end{lstlisting}
\end{itemize}

\subsection{引用}

\subsubsection*{(左值)引用}

\begin{itemize}[leftmargin=0pt, rightmargin=0cm, labelwidth=0.8cm, labelsep=0.2cm]
\item 左值引用 (通常所说的引用) 是某个变量的别名, 可以出现在赋值号的左边. 另外,
  引用不能更改它引用的对象.
\item 左值引用的声明与初始化:
\begin{lstlisting}
int i = 17;
int& r = i;              // declare a reference to variable i
int& rfive = 5;          // will not compile!
const int& rfive = 5;    // OK!
\end{lstlisting}
\item 引用与指针的差异:\\
  (1) 没有空引用, 但可以有空指针. 引用必须是参考某个实际变量.\\
  (2) 一旦引用被赋予某个对象后, 不能改变它来引用另一个对象. 而指针可以在任何时候
  指向其他任何变量.\\
  (3) 引用在创造的时候就必须初始化, 但指针可在其他时候初始化.\\
  (4) 指针必须通过解引用(使用算符 \lstinline{*})来获取它所指变量的内容, 但引用不
  需要.
\end{itemize}

\subsubsection*{右值引用}


\end{document}